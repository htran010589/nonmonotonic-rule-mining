\chapter{Conclusions and Future Work}
\label{chap:conclusion}

In this research work, we develop the RUMIS system based on the theory framework where the novel concept of \textit{partial materialization} is introduced to revise positive Horn rules. Subsequently, the chosen revisions are exploited to extend the original data, and thus, tackle the KG completion problem. Besides, some experiments are conducted for testing quality of rules generated by RUMIS and the result supports the proposed theory. In the future, there are some directions that we can develop as follows.

\begin{itemize}
\item More forms for the positive rules can be implemented, not only the form~\ref{form2} introduced in Chapter~\ref{chap:back}. This makes the RUMIS less restrictive and diversify the resulting revisions.
\item Due to large size of the KG, data indexing is a time burden step for every experiment. Thus, we can refine the RUMIS system by store the indices in a database and the rest computation may make us that via a web service subsequently.
\item Other predictive measures and exception evaluation methods can be tested to search for interesting nonmonotonic rules.
\end{itemize}