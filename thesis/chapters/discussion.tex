\chapter{Discussion}

\section{Other Approaches - related work}

Both of above-mentioned approaches are internal methods, that is, only data inside the graph is used to infer new relations. On the contrary, this section focuses on different approaches that require data outside the knowledge base, e.g, web pages linking to objects or big collection of documents.

Wikipedia pages can be used to identify relations between entities in~\cite{ref18}. In a larger scale,~\cite{ref19} proposes learning lexical predicate patterns, and searches all over the Internet to find subject-object pairs corresponding to the patterns. These new pairs can be filled to the original graph. As a result, a big text corpora is used in ~\cite{ref19} to learn relations.

With the intuition that entities appearing in the same table should have the same relations, authors in~\cite{ref20} try to refine knowledge graph based on Wikipedia tables. Similar research works are conducted using page lists~\cite{ref21} or HTML tables~\cite{ref22}.

Instead of documents, interlinks are used to add relation edges to knowledge graph~\cite{ref23, ref24}. More specifically, relations of two entities in FreeBase can be inserted to YAGO if they also appear in the latter knowledge graph. This way can be extended to mapping method with probabilities in~\cite{ref25}.