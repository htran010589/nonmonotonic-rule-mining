\chapter{Related Work}

The area of rule mining on knowledge graphs has attracted interests from many researchers recently. This problem can be classified into two main directions: statistics-based and logic-based approaches.

\section{Statistics-based Approach}

This direction focuses on building models with latent features which are not directly observable from the original data~\cite{ref1}. The core idea of this approach is to infer correlation between objects based on extracted hidden features. Besides, feature extraction is automatically executed in these methods.

RESCAL is one of principal algorithms in this direction where relations of any two hidden features are taken into consideration ~\cite{ref2, ref3, ref4}. As a result, it is called bilinear model.

\section{Logic-based Approach}

\subsection{ILP-based System}

Rule mining is a core problem in ILP-based system, however, there are two issues:

\begin{itemize}
	\item ILP system is not scalable for big data.
	\item ILP system always requires positive and negative examples while in our problem, only positive observations are given.
\end{itemize}

\subsection{AMIE+ System}

AMIE and AMIE+ are created to tackle the above-mentioned problems, but they only take care about positive rules.

\subsection{Nonmonotonic Rule Mining System}

This work is an extension from a non mototonic rule mining system (ISWC paper). The previous work focuses on flattened data while the current one treat the knowlege graph in the nature format.
