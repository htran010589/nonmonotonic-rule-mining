\chapter{Background}
\label{chap:back}

In this chapter, we introduce some preliminary knowledge for the rest of the thesis with the following organization. First, nonmonotonic logic programs under semantics answer set and KG completion problem are presented. Second, we address some definitions of association rule mining in relational setting such as \textit{head support, absolute support, confidence} and \textit{conviction}. These definitions are exploited in the main part of this work (Chapter~\ref{chap:frame} and~\ref{chap:system}).

\section{Nonmonotonic Logic Program}

In the current work, we rely on the standard definitions of logic programs~\cite{ref49}. Formally, \textit{nonmonotonic logic program P} is a rule set where each rule has the form:

\begin{equation}
r: H \leftarrow B, not E
\end{equation}
\label{rule3}

In details, $H$ stands for \textit{head(r)} which is a head of the rule $r$, i.e., a first-order atom in the format \textit{a(\textbf{X})}. Besides, $B, not E$ is a conjunction: \textit{b$_1$(\textbf{Y}$_1$), b$_2$(\textbf{Y}$_2$), ..., b$_k$(\textbf{Y}$_k$)} and \textit{not b$_{k+1}$(\textbf{Y}$_{k+1}$), b$_{k+2}$(\textbf{Y}$_{k+2}$), ..., b$_n$(\textbf{Y}$_n$)}, resp. $B$ and $not E$ are used as the short form of $body^+(r)$ and $body^-(r)$, resp. The latter is referred to \textit{negation as failure (NAF), default negation}. In case $body^-(r) = \emptyset$, $r$ is a positive (Horn) rule. \textit{\textbf{X, Y$_{1}$, Y$_{2}$, ..., Y$_{n}$}} are tuples of arguments, i.e., variables and/or constants and their sizes are the arity of \textit{a, b$_1$, b$_2$, ..., b$_n$}, resp. The signature of the program $P$ is denoted as $\Sigma_{P} = \langle$\textbf{P}$, \cC\rangle$ where \textbf{P}, $\cC$ are sets of relations and constants in $P$, resp.

A logic program $P$ is \textit{ground} if it does not contain any variables, i.e., only constants and predicates appear in each rule $r$. For a non ground program $P$, $Gr(P)$ is a ground instantiation of $P$ which is obtained by replacing variables with constants in all possible ways. The \textit{Herbrand Universe HU(P)} of $P$ is the set of constants $\cC$ appearing in program $P$ and \textit{Herbrand Base HB(P)} contains all possible ground atoms constructed by predicates in $P$ and constants in $\cC$, resp. Any subset of $HB(P)$ is a \textit{Herbrand Interpretation} of a program $P$. An interpretation $I$ is a model of a rule $r$ if for every possible substitution of variables $body^+(r), body^-(r)$ hold true, $head(r)$ is also true w.r.t. $I$. $I$ is defined as a model for a program $P$ if it is the model for all rules in $P$. In addition, $MM(P)$ denotes a set of a subset-inclusion minimal model of $P$.

An \textit{answer set} $I$ of $P$ is a Herbrand interpretation of $P$ s.t. $I \in MM(P^I)$. Here, $P^I$ denotes the Gelfond-Lifschitz (GL) reduct~\cite{ref50} of $P$ which is generated by deleting any rule $r$ s.t. $body^-(r)$ intersects with $I$ and then removing all NAFs in the rest of the rules. $AS(P)$ stands for the set of all answer sets for $P$.

\begin{example}\label{ex:as}
Consider the following nonmonotonic program as an example:\\
{\small \leftline{$P = \left\{
            \renewcommand{\arraystretch}{1.1}
            \begin{array}{@{\,}l@{~~}l@{}}
              \mbox{(1) }\mi{livesIn(brad,berlin)};\;\mbox{(2) }\mi{isMarriedTo(brad,ann)};\\
              \mbox{(3) } \mi{livesIn(Y,Z)\leftarrow isMarriedTo(X,Y),livesIn(X,Z),  \naf\ researcher(Y)}\\
            \end{array}%
            \!\right\}$}}
            
\normalsize
{\smallskip

\noindent            
We obtain the ground instantiation $Gr(P)$ of $P$ by replacing $X,Y,Z$ with $\mi{brad, \,ann}$ and $\mi{berlin}$, resp. Consider the interpretation $I=\{${\small\textit{isMarriedTo(brad,ann), livesIn\\(ann,berlin), livesIn(brad,berlin)}}$\}$. Based on the above definition, the GL-reduct $P^I$ of $P$ consists of a rule \textit{livesIn(ann,berlin) $\leftarrow$ livesIn(brad,berlin), isMarriedTo(brad,ann)} and the ground terms (1), (2). Since $I$ is a minimal model of $P^I$, by definition, we have that $I \in AS(P)$.}\qed
\end{example}

\section{Knowledge Graph Completion}

In this section, we provide a definition of KG completion based on the above concept of Answer Set Programming. The factual representation of a KG $\cG$ is the set of triples over the signature $\Sigma_{\cG}=\tuple{\mathbf{C},\mathbf{R},\mathcal{C}}$, in which $\mathbf{C}$, $\mathbf{R}$ and $\mathcal{C}$ denote the sets of unary predicates, binary predicates and constants, resp. By $G^i$, we denote a KG that consists every correct triple with predicates and constants of $\Sigma_{\cG^a}$ that holds true in the the real world. Based on~\cite{ref51}, the gap between the \emph{available graph} $\cG^a$ and $\cG^i$ is defined as follows.

\begin{definition}[Incomplete data source] A pair $G = (\cG^a, \cG^i)$ of two KGs is defined as an incomplete data source, in which $\cG^a\subseteq \cG^i$ and $\Sigma_{\cG^a}=\Sigma_{\cG^i}$.
\end{definition}

We aim to learn a nonmonotonic rule set $\cR$ from the $\cG^a$ s.t. a good approximation of $\cG^i$ is a result of applying $\cR$ to $\cG^a$, i.e., corresponds to the calculation of $AS(\cR \cup \cG)$. Now the definition of KG completion is ready to be presented as follows:

\begin{definition}[Rule-based KG completion]\label{def:graphcompl}
Given a factual representation of a KG $\cG$ over the signature $\Sigma_{\cG}=\tuple{\mathbf{C},\mathbf{R},\cC}$, a set of rules $\cR$ are mined from $\cG$ with the signature. After that, the \emph{completion of $\cG$ \wrt\ $\cR$} is defined as a graph $\cG_{\cR}$ created by any answer set in $AS(\cR \cup \cG)$.
\end{definition}

\section{Association Rule Mining in Relational Setting}

Association rule mining explores frequent patterns from the data and subsequently cast these patterns into rules. Originally, association rules were studied in the market basket context, where interesting relations between products from transaction database of customer purchases were extracted, e.g., \textit{\{onions, potatoes\} $=>$ \{burger}\}~\cite{ref54}. Recently, association rule learning methods were extended to relational settings, which attracts research interests of both ILP~\cite{ref52} and KG~\cite{ref10} scientists. In the following description, we present typical rule measures for association rules in the relational setting.

A \emph{conjunctive query} $Q$ w.r.t $\cG$ is denoted as $Q(\vec{X}):-p_1(\vec{X_1}),\dotsc,p_m(\vec{X_m})$. The body (i.e., right part) of the query is a list of positive or negative atoms over $\cG$. Meanwhile, the head (i.e., left part) is a tuple of variables in the right part. The \emph{answer} of $Q$ w.r.t $\cG$ is defined as a set $Q(\cG):=\{$substitutions $\theta$ of $\vec{X}$: $p_i(\vec{X_i}\theta) \in \cG$ with $i \in [m]\}$. Based on \cite{ref53}, the \emph{(absolute) support} of a conjunctive query $Q$ w.r.t the KG $\cG$ is the cardinality of the set $Q(\cG)$. For instance, the query
\begin{equation}\mi{Q(X,Y,Z):-isMarriedTo(X,Y),\, }\mi{livesIn(Y,Z)}
\end{equation}
has the absolute support $6$ w.r.t $\cG$ in Figure~\ref{fig1.1} because there are $6$ substitutions for triple $\langle X, Y, Z \rangle$ that fulfill \textit{isMarriedTo(X,Y), livesIn(Y,Z)} w.r.t $\cG$.

An \emph{association rule} has a format $Q_1 => Q_2$, where $Q_1$ and $Q_2$ are conjunctive queries and all atoms in the body of $Q_1$ also appear in that of $Q_2$. To be specific, from the $Q(X,Y,Z)$ in the above example and
\begin{equation}Q'(X,Y,Z):-\mi{isMarriedTo(X,Y),\,livesIn(X,Z),\,} \mi{livesIn(Y,Z)}
\end{equation} the association rule $Q => Q'$ can be built.

In the current work, association rules are exploited with the aim to predict new facts, hence, they should be translated into logical format. More specifically, we can convert association rule $Q_1=>Q_2$ to the logical one $Q_2\backslash Q_1 \leftarrow Q_1$, in which $Q_2 \backslash Q_1$ is the set of atoms which are in $Q_2$ but not in $Q_1$. For example, it can be seen that the above rule $Q=>Q'$ can be transformed to $\mi{r1}$ in Section~\ref{chap:intro}.

In this research, the conviction measure~\cite{ref48} is exploited because it is guaranteed to have high predictive power~\cite{ref46}. Hence, this measure is useful for KG completion introduced in Chapter~\ref{chap:frame}. With the rule $r:\;\mi{H\leftarrow B, \naf\ E}$, where $H=\mi{h(X,Y)}$ and $B,E$ contain a set of variables $\vec{Z}\supseteq X,Y$, we can find the \emph{conviction} by the following formula:
\vspace{-.26cm}
\begin{equation}
\mi{conv(r, \cG)= \dfrac{1 - supp(h(X,Y), \cG)}{1 - conf(r, \cG)}}
\end{equation}
in which $\mi{supp(h(X,Y),\cG)}$ stands for \textit{relative support} of the head $\mi{h(X,Y)}$, it is calculated as:
\vspace{-.28cm}
\begin{equation}
supp(h(X,Y),\cG)=\dfrac{\#(X,Y):h(X,Y)\in \cG}{(\#X:\exists Y\;h(X,Y)\in \cG)*(\#Y:\exists X\;h(X,Y)\in \cG)}
\end{equation}
besides, $\mi{conf}$ denotes the \textit{confidence} of $r$:
\begin{equation}
\mi{conf(r,\cG)=\dfrac{\#(X,Y): H \in \cG, \exists \vec{Z}\;B\in \cG,E \not \in \cG}{\#(X,Y):\exists \vec{Z}\; B\in \cG, E \not \in \cG}}
\end{equation}
\vspace{-.3cm}

\begin{example}
Applying these definitions to the KG in Figure~\ref{fig1.1}, we gain some statistics as follows. Given a KG $\cG$ in Figure~\ref{fig1.1} and a rule $r1$ in the form~\ref{rule1}, there are three ways to substitute variables in $r1$ based on $\cG$. Thus, the absolute support of $r1$ w.r.t. $\cG$ is $3$. Besides, the \textit{relative head support} of predicates \textit{livesIn} and \textit{hasFriend} are $\mi{supp(livesIn(X,Z)), \cG}{=}\dfrac{9}{8 \times 4}{\approx} 0.3$ and  $\mi{supp(hasFriend(X,Z)), \cG}{=}\dfrac{1}{1 \times 1}{=} 1.0$, resp. The confidence of rule $r1$ is $conf(r1, \cG) {=} \dfrac{2}{5} {=} 0.4$, thus, the conviction measure can be found as $conv(r1, \cG) {=} \dfrac{1-0.3}{1 - 0.4} {=} \dfrac{7}{6} {=} 1.17$
\end{example}