\chapter{Background}
\label{chap:back}

In this chapter, we introduce some preliminary knowledge for the rest of the thesis with the following organization. First, nonmonotonic logic programs under answer set semantics and KG completion problem are presented. Second, we address some definitions of relational association rule mining such as \textit{head support, absolute support, confidence} and \textit{conviction}. These definitions are exploited in the main part of this work (Chapter~\ref{chap:frame} and~\ref{chap:system}).

\section{Nonmonotonic Logic Program}

In the current work, we rely on the standard definitions of logic programs~\cite{ref49}. Formally, \textit{nonmonotonic logic program P} is a ruleset where each rule has the form:

\begin{equation}
r: H \leftarrow B, \naf\ E
\end{equation}
\label{rule3}

In details, $H$ stands for \textit{head(r)} which is the head of the rule $r$, i.e., a first-order atom in the format \textit{a(\textbf{X})}. Besides, \textit{B, not E} is a conjunction: \textit{b$_1$(\textbf{Y}$_1$), b$_2$(\textbf{Y}$_2$), ..., b$_k$(\textbf{Y}$_k$)} and \textit{not b$_{k+1}$(\textbf{Y}$_{k+1}$), not b$_{k+2}$(\textbf{Y}$_{k+2}$), ..., not b$_n$(\textbf{Y}$_n$)}, respectively. $B$ and \textit{not E} are used as a short form of $body^+(r)$ and $body^-(r)$, respectively. The $not$ operator is called the \textit{negation as failure (NAF) or default negation}. If $body^-(r) = \emptyset$ then $r$ is a positive (Horn) rule. \textit{\textbf{X, Y$_{1}$, Y$_{2}$, ..., Y$_{n}$}} are tuples of arguments, i.e., variables and/or constants and their sizes are the arity of \textit{a, b$_1$, b$_2$, ..., b$_n$}, respectively. The signature of the program $P$ is denoted as $\Sigma_{P} = \langle$\textbf{P}$, \cC\rangle$ where \textbf{P}, $\cC$ are the sets of predicates and constants in $P$, respectively.

A logic program $P$ is \textit{ground} if it does not contain any variables, i.e., only constants and predicates can appear in each rule $r$. For a non ground program $P$, $Gr(P)$ is a ground instantiation of $P$ which is obtained by replacing variables with constants in all possible ways. The \textit{Herbrand Universe HU(P)} of $P$ is the set of constants $\cC$ appearing in the program $P$ and \textit{Herbrand Base HB(P)} contains all possible ground atoms constructed by predicates in $P$ and constants in $\cC$, respectively. Any subset of $HB(P)$ is a \textit{Herbrand Interpretation} of a program $P$. An interpretation $I$ is a model of a rule $r$ if for every possible substitution of variables with constants for which $body^+(r), body^-(r)$ are true, $head(r)$ is also true w.r.t. $I$. $I$ is defined as a model of a program $P$ if it satisfies all rules in $P$. In addition, $MM(P)$ denotes a set of a subset-inclusion minimal models of $P$.

An \textit{answer set} $I$ of $P$ is a Herbrand interpretation of $P$ s.t. $I \in MM(P^I)$. Here, $P^I$ denotes the Gelfond-Lifschitz (GL) reduct~\cite{ref50} of $P$ which is generated by deleting any rule $r$ s.t. $body^-(r)$ intersects with $I$ and then removing all NAFs in the rest of the rules. $AS(P)$ stands for the set of all answer sets of $P$.

\begin{example}\label{ex:as}
Consider the following nonmonotonic program as an example:\\
{\small \leftline{$P = \left\{
            \renewcommand{\arraystretch}{1.1}
            \begin{array}{@{\,}l@{~~}l@{}}
              \mbox{(1) }\mi{livesIn(brad,berlin)};\;\mbox{(2) }\mi{isMarriedTo(brad,ann)};\\
              \mbox{(3) } \mi{livesIn(Y,Z)\leftarrow isMarriedTo(X,Y),livesIn(X,Z),  \naf\ researcher(Y)}\\
            \end{array}%
            \!\right\}$}}

\normalsize
{\smallskip

\noindent            
We obtain the ground instantiation $Gr(P)$ of $P$ by replacing $X,Y,Z$ with $\mi{brad, \,ann}$ and $\mi{berlin}$, respectively. Consider the interpretation $I=\{${\small\textit{isMarriedTo(brad, ann), livesIn(ann, berlin), livesIn(brad, berlin)}}$\}$. Based on the above definition, the GL-reduct $P^I$ of $P$ consists of a rule \textit{livesIn(ann,berlin) $\leftarrow$ livesIn(brad,berlin), isMarriedTo(brad,ann)} and the ground terms (1), (2). Since $I$ is a minimal model of $P^I$, by definition, we have that $I \in AS(P)$.}\qed
\end{example}

\section{Knowledge Graph Completion}

In this section, we provide a definition of a KG completion problem based on the above concept of Answer Set Programming. The factual representation of a KG $\cG$ is the set of ground atoms over the signature $\Sigma_{\cG}=\tuple{\mathbf{C},\mathbf{R},\mathcal{C}}$, in which $\mathbf{C}$, $\mathbf{R}$ and $\mathcal{C}$ denote the sets of unary predicates, binary predicates and constants, respectively. By $\cG^i$, we denote a KG that contains all correct facts with predicates and constants from $\Sigma_{\cG^a}$ that are true in the real world. Based on~\cite{ref51}, the gap between the \emph{available graph} $\cG^a$ and $\cG^i$ is defined as follows.

\begin{definition}[Incomplete data source] A pair $G = (\cG^a, \cG^i)$ of two KGs is an incomplete data source, where $\cG^a\subseteq \cG^i$ and $\Sigma_{\cG^a}=\Sigma_{\cG^i}$.
\end{definition}

We aim at learning a nonmonotonic ruleset $\cR$ from $\cG^a$ s.t. application of $\cR$ to $\cG^a$ results in a good approximation of $\cG^i$. Application of $\cR$ to $\cG^a$ corresponds to the calculation of $I \in AS(\cR \cup \cG)$. More specifically, we define the rule based KG completion as follows:

\begin{definition}[Rule-based KG completion]\label{def:graphcompl}
Given a factual representation of a KG $\cG$ over the signature $\Sigma_{\cG}=\tuple{\mathbf{C},\mathbf{R},\cC}$ and a set of rules $\cR$ mined from $\cG$. The \emph{completion of $\cG$ \wrt\ $\cR$} is a graph $\cG_{\cR}$ constructed from any answer set in $AS(\cR \cup \cG)$.
\end{definition}

\section{Association Rule Mining in Relational Setting}

Association rule mining concerns the extraction of frequent patterns from the data and their subsequent casting into rules. Originally, association rules were studied in the market basket context, where interesting relations between products from a transaction database of customer purchases were extracted, e.g., \textit{\{onions, potatoes\} $=>$ \{burger}\}~\cite{ref54}. Recently, association rule learning methods were extended to relational settings, which attracts research interests of both ILP~\cite{ref52} and KG~\cite{ref10} scientists. In the following description, we present typical rule measures for association rules in the relational setting.

A \emph{conjunctive query} $Q$ w.r.t $\cG$ is the expression of the form $Q(\vec{X}):-\ p_1(\vec{X_1}), \dotsc,$ $p_m(\vec{X_m})$. The body (i.e., right part) of the query is a list of positive or negative atoms over $\cG$. The head (i.e., left part) is a tuple of variables appearing in the right part. The \emph{answer} of $Q$ w.r.t $\cG$ is defined as a set $Q(\cG):=\{$substitutions $\theta$ of $\vec{X}$: $p_i(\vec{X_i}\theta) \in \cG$ with $i \in [1..m]\}$. Based on \cite{ref53}, the \emph{(absolute) support} of a conjunctive query $Q$ w.r.t the KG $\cG$ is the cardinality of the set $Q(\cG)$. For instance, the query
\begin{equation}\mi{Q(X,Y,Z)\ :-\ isMarriedTo(X,Y),\, }\mi{livesIn(Y,Z)}
\end{equation}
has the absolute support $6$ w.r.t $\cG$ in Figure~\ref{fig1.1}, since there are $6$ substitutions for the triple $\langle X, Y, Z \rangle$ that satisfy \textit{isMarriedTo(X,Y), livesIn(Y,Z)} w.r.t $\cG$.

An \emph{association rule} is of the format $Q_1 => Q_2$, where $Q_1$ and $Q_2$ are conjunctive queries and all atoms in the body of $Q_1$ also appear in that of $Q_2$. For instance, from $Q(X,Y,Z)$ in the above example and
\begin{equation}Q'(X,Y,Z)\ :-\ \mi{isMarriedTo(X,Y),\,livesIn(X,Z),\,} \mi{livesIn(Y,Z)}
\end{equation} the association rule $Q => Q'$ can be built.

In the current work, association rules are exploited with the aim to predict new facts, hence, they should be translated into logical format. More specifically, we convert the association rule $Q_1=>Q_2$ to the logical one $Q_2\backslash Q_1 \leftarrow Q_1$, where $Q_2 \backslash Q_1$ is the set of atoms which are in $Q_2$ but not in $Q_1$. For example, it can be seen that the above rule $Q=>Q'$ can be transformed to $\mi{r1}$ in Section~\ref{chap:intro}.

In this work, the conviction measure~\cite{ref48} is exploited for estimating the rule quality, since it is guaranteed to have high predictive power~\cite{ref46}. Hence, this measure is useful for KG completion introduced in Chapter~\ref{chap:frame}. With the rule $r:\;\mi{H\leftarrow B, \naf\ E}$, where $H=\mi{h(X,Y)}$ and $B,E$ contain a set of variables $\vec{Z}\supseteq X,Y$, we can find the \emph{conviction} by the following formula:
\vspace{-.26cm}
\begin{equation}
\mi{conv(r, \cG)= \dfrac{1 - supp(h(X,Y), \cG)}{1 - conf(r, \cG)}}
\end{equation}
in which $\mi{supp(h(X,Y),\cG)}$ stands for \textit{relative support} of the head $\mi{h(X,Y)}$, defined as:
\vspace{-.28cm}
\begin{equation}
supp(h(X,Y),\cG)=\dfrac{\#(X,Y):h(X,Y)\in \cG}{(\#X:\exists Y\;h(X,Y)\in \cG)*(\#Y:\exists X\;h(X,Y)\in \cG)}
\end{equation}
besides, $\mi{conf}$ denotes the \textit{confidence} of $r$, given as:
\begin{equation}
conf(r,\cG)=\dfrac{\#(X,Y): H \in \cG, \exists \vec{Z}\;B\in \cG,E \not \in \cG}{\#(X,Y):\exists \vec{Z}\; B\in \cG, E \not \in \cG}
\end{equation}
\vspace{-.3cm}

\begin{example}
Applying these definitions to the KG in Figure~\ref{fig1.1}, we obtain the following results. Given a KG $\cG$ in Figure~\ref{fig1.1} and a rule $\mi{r1}$ in the form~\ref{rule1}, there are three ways to substitute variables in $\mi{r1}$ based on $\cG$. Thus, the absolute support of $\mi{r1}$ w.r.t. $\cG$ is $3$. Besides, the \textit{relative head support} of predicates \textit{livesIn} and \textit{hasFriend} are $\mi{supp(livesIn(X,Z), \cG)}{=}\dfrac{9}{8 \times 4}{\approx} 0.3$ and  $\mi{supp(hasFriend(X,Z), \cG)}{=}\dfrac{1}{1 \times 1}{=} 1.0$, respectiv-ely. The confidence of the rule $\mi{r1}$ is $conf(\mi{r1}, \cG) {=} \dfrac{2}{5} {=} 0.4$, thus, the conviction measure is as follows $conv(\mi{r1}, \cG) {=} \dfrac{1-0.3}{1 - 0.4} {=} \dfrac{7}{6} {=} 1.17$
\end{example}